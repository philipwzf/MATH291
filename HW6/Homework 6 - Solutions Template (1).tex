\documentclass[11pt,letterpaper,cm]{nupset}
\usepackage[margin=1in]{geometry}
\usepackage{graphicx}
\usepackage{amsmath,amssymb,amsthm,amscd,graphicx,wasysym,enumerate}
\usepackage{mathrsfs}

\newtheorem{theorem}{Theorem}

\newcommand{\mb}[1]{\boldsymbol{#1}}
\newcommand{\bvec}[1]{\left[\begin{smallmatrix} #1 \end{smallmatrix}\right]}
\newcommand{\bmat}[1]{\begin{bmatrix} #1 \end{bmatrix}}

% info for header block in upper right hand corner
\name{Solutions}
\class{Math 291-3}
\assignment{Homework 6}
\duedate{May 16, 2022}

\begin{document}

\begin{problem}[Exercise 1] Suppose $T: \mathbb{R}^n \to \mathbb{R}^n$ is $C^1$. Denote the coordinates of the domain by $(u_1,\ldots,u_n)$ the coordinates of the codomain by $(x_1,\ldots,x_n)$, and the components of $T$ by
	$$T(u_1,\ldots,u_n) = (x_1(u_1,\ldots,u_n),\ldots,x_n(u_1,\ldots,u_n)).$$
	Show that
	$$T^\ast(dx_1\wedge \cdots \wedge dx_n) = (\det DT(u_1,\ldots,u_n))\,du_1\wedge \cdots \wedge du_n.$$
	Hint: This is most easily done with the pattern characterization of the determinant.
\end{problem}
\begin{solution}
	By the distributive property of pullbacks, we have
	$$T^\ast(dx_1\wedge \cdots \wedge dx_n)=(T^\ast dx_1)\wedge\cdots\wedge(T^\ast dx_n)=d(\frac{\partial x_1}{\partial u_1}+\cdots+\frac{\partial x_1}{\partial u_n})\wedge\cdots\wedge d(\frac{\partial x_n}{\partial u_1}+\cdots+\frac{\partial x_n}{\partial u_n})$$
	Since if $du_i\wedge du_j=0$ for $i=j$, so the above expression is equivalent as having an $n\times n$ patterns with $n$ pairs of indices $P=\{(1,i_1),\ldots,(n,i_n)\}$ such that $i_1,\ldots,i_n$ is a rearrangement of $1,\ldots,n$ with no repeats. Then, we have
	$$T^\ast(dx_1\wedge \cdots \wedge dx_n)=\sum_{\{i_1,\ldots,i_n\}=\{1,\ldots,n\}} d(\frac{\partial x_1}{\partial u_{i_1}})\wedge\cdots\wedge d(\frac{\partial x_n}{\partial u_{i_n}})$$
	Now, we will do a swap on each of the term $d(\frac{\partial x_p}{\partial u_{i_p}})$ and $d(\frac{\partial x_q}{\partial u_{i_q}})$ if $p<q,i_p>i_q$. Define $c$ to be the number of swaps. With each swap of the terms, we need to multiply the term by $(-1)$. Then this is equivalent to the signature of the pattern $P$. Then,
	\begin{align*}
		T^\ast(dx_1\wedge \cdots \wedge dx_n)&=\sum_{\{i_1',\ldots,i_n'\}=\{1,\ldots,n\}} d(\frac{\partial x_{i_1'}}{\partial u_{1}})\wedge\cdots\wedge d(\frac{\partial x_{i_n'}}{\partial u_n})(sgn(P))\\
		&=\sum_{\{i_1',\ldots,i_n'\}=\{1,\ldots,n\}} (\frac{\partial x_{i_1'}}{\partial u_{1}}\cdots\frac{\partial x_{i_n'}}{\partial u_n})(sgn(P))du_1\wedge\cdots\wedge du_n\\
		&=\sum_{n\times n\ \mathrm{ patterns P}} (\mathrm{prod}\ P(DT))(\mathrm{sgn}\ (P))du_1\wedge\cdots\wedge du_n\\
		&=(det DT(u_1,\ldots,u_n))du_1\wedge\cdots\wedge du_n
	\end{align*}

\end{solution}
\newpage

\begin{problem}[Exercise 2] A $k$-form $\omega$ on $\mathbb{R}^n$ is called \textbf{closed} if $d\omega = 0$, and is called \textbf{exact} if there exists a $(k-1)$-form $\alpha$ on $\mathbb{R}^n$ such that $d\alpha = \omega$. Show that every exact $C^1$ $k$-form is closed. In other words, this is asking to show that
	$$d^2\alpha \stackrel{def}{=} d(d\alpha) \text{ is zero}$$
	for every $C^2$ $(k-1)$-form alpha. 
	\medskip
	
	Hint: Since $d$ is linear, it is enough to check that applying $d$ twice to something of the form
	$$f(x_1,\ldots,x_n)\,dx_{i_1}\wedge\cdots\wedge dx_{i_k},$$
	where $f$ is $C^2$, results in zero. Clairaut's Theorem is important here.
\end{problem}
\begin{solution}
	pf: We will show that 
	$$d^2(f(x_1,\ldots,x_n)\,dx_{i_1}\wedge\cdots\wedge dx_{i_k})=0$$
	Let $dx_{i_{k+1}},\ldots,dx_{i_n}$ be the other variables in $f$ but not in the original $k$-form. Since if there is a repeting term in the wedge product, the whole term would be equal to zero, so we will simplify the expression by omitting those terms. (For example, $f_{x_1}dx_{i_1}\wedge dx_{i_1}\wedge\cdots\wedge dx_{i_k}=0$ will be omitted) Then,
	\begin{align*}
		d^2(f(x_1,\ldots,x_n)\,dx_{i_1}\wedge\cdots\wedge dx_{i_k})&=d(df(x_1,\ldots,x_n)\,dx_{i_1}\wedge\cdots\wedge dx_{i_k})\\
		&=d((f_{x_1}dx_1+\cdots+f_{x_n}dx_n)dx_{i_1}\wedge\cdots\wedge dx_{i_k})\\
		&=d(f_{x_{i_{k+1}}}dx_{i_{k+1}}\wedge dx_{i_1}\wedge\cdots\wedge dx_{i_k}+\cdots+f_{x_{i_n}}dx_{i_n}\wedge dx_{i_1}\wedge\cdots\wedge dx_{i_k})\\
		&=d(f_{x_{i_{k+1}}}dx_{i_{k+1}}\wedge dx_{i_1}\wedge\cdots\wedge dx_{i_k})+\cdots+d(f_{x_{i_n}}dx_{i_n}\wedge dx_{i_1}\wedge\cdots\wedge dx_{i_k})\\
		&=(f_{x_{i_{k+1}}x_1}dx_1+\cdots+f_{x_{i_{k+1}}x_n}dx_n) dx_{i_{k+1}}\wedge dx_{i_1}\wedge\cdots\wedge dx_{i_k}\\
		&+\cdots+(f_{x_{i_{n}}x_1}dx_1+\cdots+f_{x_{i_{n}}x_n}dx_n) dx_{i_{n}}\wedge dx_{i_1}\wedge\cdots\wedge dx_{i_k}\\
		&=((f_{x_{i_{k+1}}x_1}dx_1+\cdots+f_{x_{i_{k+1}}x_n}dx_n) dx_{i_{k+1}}+\cdots+(f_{x_{i_{n}}x_1}dx_1+\\
		&\cdots+f_{x_{i_{n}}x_n}dx_n) dx_{i_{n}}))dx_{i_1}\wedge\cdots\wedge dx_{i_k}
	\end{align*}
	Now consider all the terms inside the parenthesis. All the $dx_{i_1},\ldots dx_{i_k}$ will be zero after getting the wedge product with the outside. And the double derivatives regarding the same variable will also cancel out since they will also have the wedge product with the same variable. So the only terms left can be represented as $f_{x_px_q}dx_p\wedge dx_q$ and $f_{x_qx_p}dx_q\wedge dx_p$ where $p,q\in\{x_{i_{k+1}},\ldots,x_{i_n}\}$ and $p\neq q$. Since $f$ is $C^2$, by Clairaut's Theorem,
	$$f_{x_qx_p}dx_q\wedge dx_p=f_{x_px_q}dx_q\wedge dx_p=-f_{x_px_q}dx_p\wedge dx_q$$
	So all the remaining terms in the parenthesis will also cancel out. Thus,
	$$d^2(f(x_1,\ldots,x_n)\,dx_{i_1}\wedge\cdots\wedge dx_{i_k})=0dx_{i_1}\wedge\cdots\wedge dx_{i_k}=0$$
	Thus we have shown if there exists a $k$-form $\alpha$ such that $d\alpha$ is a $(k+1)$-form, then $d^2\alpha=0$. 
\end{solution}
\newpage

\begin{problem}[Exercise 3] Which of the following differential forms on $\mathbb{R}^3$ are closed?
	
	\begin{itemize}
		\item[(a)] $e^{xyz}\,dy\wedge dz\wedge dx$
		\item[(b)] $xy^2\,dx - zx\,dy+y\,dz$
		\item[(c)] $(x^2y+e^z)\,dx\wedge dy + (e^{xy}-xe^z)\,dy\wedge dz + ((x^2+1)^{\sin(xz)}+2)\,dz\wedge dx$
	\end{itemize}
\end{problem}
\begin{solution}
\end{solution}
\newpage

\begin{problem}[Exercise 4] This problem has two parts.
	\begin{itemize}
		\item[(a)] Show that the differential form
		$$\omega = \frac{-y\,dx+x\,dy}{x^2+y^2}$$
		is closed on the region $U=\mathbb{R}^2-\{(0,0)\}$ obtained by removing the origin from $\mathbb{R}^2$. 
		\item[(b)] The form in (a) is not exact on $U$, but show that it is exact on the region obtained by removing the positive $x$-axis from $U$. In other words, find a $C^1$ function on this region whose differential is $\omega$. (This is essentially the same as Problem 10 on Homework 5, only in that case the question was phrased in terms of vector fields and here it is phrased in terms of differential forms. Nonetheless, the steps outlined there and in the solution to that problem are the ones you need here as well.)
	\end{itemize}
\end{problem}
\begin{solution}
\end{solution}
\newpage

\begin{problem}[Exercise 5] Show that the differential $1$-form
	$$\frac{x\,dx+y\,dy+z\,dz}{x^2+y^2+z^2}$$
	defined on $\mathbb{R}^3-\{(0,0,0)\}$ is closed and exact.
\end{problem}
\begin{solution}
	pf: Let $f=\frac{ln(x^2+y^2+z^2)}{2}$ be the 0-form. Then, on $\mathbb{R}^3-\{(0,0,0)\}$, 
	$$df=d\left(\frac{ln(x^2+y^2+z^2)}{2}\right)=\frac{x}{x^2+y^2+z^2}dx+\frac{y}{x^2+y^2+z^2}dy+\frac{z}{x^2+y^2+z^2}dz=\frac{x\,dx+y\,dy+z\,dz}{x^2+y^2+z^2}$$
	Now we will show that it is closed.
	\begin{align*}
		d\left(\frac{x\,dx+y\,dy+z\,dz}{x^2+y^2+z^2}\right)&=d\left(\frac{x}{x^2+y^2+z^2}dx+\frac{y}{x^2+y^2+z^2}dy+\frac{z}{x^2+y^2+z^2}dz\right)\\
		&=\left(\frac{-2xy}{(x^2+y^2+z^2)^2}dy+\frac{-2xz}{(x^2+y^2+z^2)^2}dz\right)dx+\\
		&\hspace*{7mm}\left(\frac{-2yx}{(x^2+y^2+z^2)^2}dx+\frac{-2yz}{(x^2+y^2+z^2)^2}dz\right)dy+\\
		&\hspace*{9mm}\left(\frac{-2zx}{(x^2+y^2+z^2)^2}dx+\frac{-2zy}{(x^2+y^2+z^2)^2}dy\right)dz\\
		&=\frac{(-2xy)dy\wedge dx-(2xz)dz\wedge dx-(2yx)dx\wedge dy-(2yz)dz\wedge dy-(2zx)dx\wedge dz-(2zy)dy\wedge dz}{(x^2+y^2+z^2)^2}\\
		&=0
	\end{align*}
	So it is closed. 
\end{solution}
\newpage

\begin{problem}[Exercise 6] (Colley 6.1.15, 6.1.19) This problem has two parts.
	
	\begin{itemize}
		\item[(a)] Find $\displaystyle\int_{\vec{x}} \vec{F} \cdot d\vec{s}$ where
		$$\vec{F}(x,y,z) = 3z\,\vec{i}+y^2\,\vec{j}+6z\,\vec{k} \quad\text{and}\quad \vec{x}(t) = (\cos(t), \sin(t), t/3),\ 0 \le t \le 4\pi.$$
		\item[(b)] If $\vec{x}(t) = (e^{2t}\cos 3t,e^{2t}\sin 3t), 0 \le t \le 2\pi$, find
		$$\int_{\vec{x}} \frac{x\,dx+y\,dy}{(x^2+y^2)^{3/2}}$$
	\end{itemize}
\end{problem}
\begin{solution}
\end{solution}
\newpage
 
\begin{problem}[Exercise 7] (Colley 6.1.23, 6.1.32) This problem has two parts.
	\begin{itemize}
		\item[(a)] Let $\vec{F}(x,y,z) = (2z^5-3yx)\,\vec{i}-x^2\,\vec{j}+x^2z\,\vec{k}$. Calculate the line integral of $\vec{F}$ around the perimeter of the square with vertices $(1,1,3), (-1,1,3), (-1,-1,3), (1,-1,3)$, oriented counterclockwise about the $z$-axis as viewed from the positive $z$-axis.
		\item[(b)] Calculate $\displaystyle\int_C z\,dx+x\,dy+y\,dz$ where $C$ is the curve obtained by intersecting the surfaces $z=x^2$ and $x^2+y^2=4$ and is oriented counterclockwise around the $z$-axis as seen from the positive $z$-axis.
	\end{itemize}
\end{problem}
\begin{solution}
\end{solution}
\newpage

\begin{problem}[Exercise 8] (Colley 6.1.33) Show that $\displaystyle\int_{\vec{x}} \vec{T}\cdot d\vec{s}$ equals the length of the path $\vec{x}$, where $\vec{T}$ denotes the unit tangent vector of the path.
\end{problem}
\begin{solution}
\end{solution}
\newpage

\begin{problem}[Exercise 9] (Colley 6.1.37) Suppose $C$ is the curve $y=f(x)$, oriented from $(a,f(a))$ to $(b,f(b))$ where $a < b$ and where $f$ is positive and $C^1$ on $[a,b]$. If $\vec{F} = y\,\vec{i}$, show that the value of $\displaystyle\int_C \vec{F}\cdot d\vec{s}$ is the area under the graph of $f$ between $x=a$ and $x=b$.
\end{problem}
\begin{solution}
\end{solution}
\newpage

\begin{problem}[Exercise 10] (Colley 6.1.38, 6.1.39) This problem has two parts.
	\begin{itemize}
		\item[(a)] Let $\vec{F}$ be the radial vector field $\vec{F}(x,y,z) = x\,\vec{i}+y\,\vec{j}+z\,\vec{k}$. Show that if $\vec{x}(t), a \le t \le b$ is any path that lies on the sphere $x^2+y^2+z^2=c^2$, then $\displaystyle\int_{\vec{x}} \vec{F} \cdot d\vec{s} = 0$.
		\item[(b)] Suppose that $C$ is a smooth oriented curve that lies in the level set of a $C^1$ function \newline $f:\mathbb{R}^2\to\mathbb{R}$.  Show that $\displaystyle\int_C \nabla f \cdot d\vec{s} = 0$ .
	\end{itemize}
\end{problem}
\begin{solution}
\end{solution}

\end{document}
