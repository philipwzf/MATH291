\documentclass[11pt,letterpaper,cm]{nupset}
\usepackage[margin=1in]{geometry}
\usepackage{graphicx}
\usepackage{amsmath,amssymb,amsthm,amscd,graphicx,wasysym,enumerate}
\usepackage{mathrsfs}

\newtheorem{theorem}{Theorem}

\newcommand{\mb}[1]{\boldsymbol{#1}}
\newcommand{\bvec}[1]{\left[\begin{smallmatrix} #1 \end{smallmatrix}\right]}
\newcommand{\bmat}[1]{\begin{bmatrix} #1 \end{bmatrix}}

% info for header block in upper right hand corner
\name{Solutions}
\class{Math 291-3}
\assignment{Homework 5}
\duedate{May 6, 2022}

\begin{document}

\begin{problem}[Exercise 1]  (Colley 7.1.27, 7.1.29) For each part, compute the surface area of the given surface $S\subset\mathbb{R}^3$.
	\begin{itemize}
		\item[(a)] $S$ is the surface cut from the paraboloid $z=2x^2+2y^2$ by the planes $z=2$ and $z=8$.
		\item[(b)] $S$ is the surface defined by the equation $z=f(x,y)$ for a $C^1$ function $f:D\to\mathbb{R}$ over an elementary region $D\subset\mathbb{R}^2$, assuming that there is $a\in(0,\infty)$ such that for each $(x,y)\in D$, $(f_x(x,y))^2+(f_y(x,y))^2=a$. Your answer should be in terms of the area of $D$.
	\end{itemize}
\end{problem}
\begin{solution}
	\begin{itemize}
		\item[(a)] 
		\item[(b)] 
	\end{itemize}
\end{solution}
\newpage

\begin{problem}[Exercise 2] (Colley 7.1.30) Let $S$ be the surface defined by
	$$z=\frac1{\sqrt{x^2+y^2}} \quad\text{for } z \ge 1.$$
	The point of this problem is to show that $S$ has infinite surface area, but nevertheless encloses a region of finite volume.
	\begin{itemize}
		\item[(a)] Sketch the graph of this surface.
		\item[(b)] Show that the volume of the region bounded by $S$ and the plane $z=1$ is finite. (You can compute this as an improper integral. To do so, let $W_a$ denote the region bounded above by $S$, below by $z=1$, and that lies outside of the cylinder $x^2+y^2=a^2$.  Then the volume you are after should be given as $\lim\limits_{a\to 0+} {\rm Vol}_3(W_a)$.  Compute ${\rm Vol}_3(W_a)$, and then compute the limit using single-variable calculus techniques.)
		\item[(c)] Show that the surface area of $S$ is infinite. (You will need to use another improper integral here.)
	\end{itemize}
\end{problem}
\begin{solution}
		\begin{itemize}
		\item[(a)] 
		\item[(b)] 
		\item[(c)] 
	\end{itemize}
\end{solution}
\newpage

\begin{problem}[Exercise 3] (Colley 7.1.32, 7.1.33) In this problem you determine formulas for the surface area of graphs of functions defined in terms of cylindrical and spherical coordinates.
	\begin{itemize}
		\item[(a)] Suppose that a surface $S$ is given in cylindrical coordinates by the equation $z=f(r,\theta)$, where $(r,\theta)$ varies through a region $D$ in the $r\theta$-plane where $r$ is nonnegative. Show that the surface area of $S$ is given by
		$$\iint_D \sqrt{1+\left(f_r(r,\theta)\right)^2+\frac1{r^2}\left(f_\theta(r,\theta)\right)^2}\,r\,dA(r,\theta).$$
		\item[(b)] Suppose that a surface $S$ is given in spherical coordinates by the equation $\rho=f(\phi,\theta)$, where $(\phi,\theta)$ varies through a region $D$ in the $\phi\theta$-plane and $f(\phi,\theta)$ is nonnegative. Show that the surface area of $S$ is given by
		$$\iint_D f(\phi,\theta)\sqrt{((f(\phi,\theta))^2+(f_\phi(\phi,\theta))^2)\sin^2\phi+(f_\theta(\phi,\theta))^2}\,dA(\phi,\theta).$$
	\end{itemize}
\end{problem}
\begin{solution}
	\begin{itemize}
		\item[(a)]
		\item[(b)]
	\end{itemize}
\end{solution}
\newpage

\begin{problem}[Exercise 4] Suppose $S \subset \mathbb{R}^3$ is a smooth $C^1$ surface with two parametrizations:
	$$\vec{X}: D \to \mathbb{R}^3 \quad\text{and}\quad \vec{Y}: E \to \mathbb{R}^3$$
	where $D$ is a region in the $uv$-plane and $E$ a region in the $st$-plane. To be clear, $u$ and $v$ denote the parameters in the $\vec{X}$-parametric equations and $\vec{X}(u,v)$ is defined for $(u,v)\in D$, and $s$ and $t$ denote the parameters in the $\vec{Y}$-parametric equations and $\vec{Y}(s,t)$ is defined for $(s,t)\in E$. Suppose further that these two parametrizations are related by $\vec{Y}(s,t) = \vec{X} \circ T(s,t)$ where $T: E \to D$ is some $C^1$ bijective map with ${\rm det} DT(s,t)\neq 0$ for every $(s,t)\in E$.
	\begin{itemize}
		\item[(a)] (This part is only to establish the standard algebraic properties of the cross product, which will be useful in part (b) and going forward.)  Recall from Problem 3 on Homework 3 from MATH 291-2, for $\vec{u},\vec{v}\in \mathbb{R}^3$ the cross product $\vec{u}\times\vec{v}\in\mathbb{R}^3$ is the unique vector that satisfies $\vec{x}\cdot(\vec{u}\times\vec{v})={\rm det}(\vec{x},\vec{u},\vec{v})$ for every $\vec{x}\in\mathbb{R}^3$.  Prove that for every $\vec{u},\vec{v},\vec{w}\in\mathbb{R}^3$ and $\lambda\in\mathbb{R}$, 
		\begin{itemize}
			\item[(i)] $\vec{u}\times\vec{v}=-(\vec{v}\times\vec{u})$,
			\item[(ii)] $\vec{u}\times(\vec{v}+\vec{w})=\vec{u}\times\vec{v}+\vec{u}\times\vec{w}$ and $(\vec{u}+\vec{v})\times\vec{w}=\vec{u}\times\vec{w}+\vec{v}\times\vec{w}$, and
			\item[(iii)] $(\lambda \vec{u})\times \vec{v}=\lambda(\vec{u}\times\vec{v})=\vec{u}\times(\lambda\vec{v}).$
		\end{itemize}
		\item[(b)] If $N_{\vec{X}}(u,v)$ and $N_{\vec{Y}}(s,t)$ denote the normal vectors arising from $\vec{X}$ and $\vec{Y}$ respectively at a point $(u,v) = T(s,t)$, show that
		$$N_{\vec{Y}}(s,t) = (\det DT(s,t))N_{\vec{X}}(T(s,t)).$$
		This says that the Jacobian $\det DT(s,t)$ describes how to express normal vectors with respect to one parametrization in terms of normal vectors with respect to another parametrization.  (Suggestion: First apply the chain rule to $\vec{Y}(s,t)=\vec{X}(T(s,t))$, and then use the result to write $\vec{Y}_s(s,t)$ and $\vec{Y}_t(s,t)$ as linear combinations of $\vec{X}_u(T(s,t))$ and $\vec{X}_v(T(s,t))$.)
		\item[(c)] Show that
		$$\iint_E \|N_{\vec{Y}}(s,t)\|\,dA(s,t) = \iint_D \|N_{\vec{X}}(u,v)\|\,dA(u,v),$$
		and therefore the surface area of $S$ is well-defined (i.e. it does not depend on which parametrization for $S$ we use).
	\end{itemize}
\end{problem}
\begin{solution}
	\begin{itemize}
		\item[(a)]Let $\vec{x}\in\mathbb{R}^3$. Then,
		\begin{itemize}
			\item[(i)] $$\vec{x}\cdot(\vec{u}\times\vec{v})=det(\vec{x},\vec{u},\vec{v})=-det(\vec{x},\vec{v},\vec{u})=\vec{x}\cdot(-(\vec{v}\times\vec{u}))$$
			By the cancellation property of dot product, $\vec{u}\times\vec{v}=-(\vec{v}\times\vec{u})$.
			\item[(ii)] $$\vec{x}\cdot(\vec{u}\times(\vec{v}+\vec{w}))=det(\vec{x},\vec{u},\vec{v}+\vec{w})=det(\vec{x},\vec{u},\vec{v})+det(\vec{x},\vec{u},\vec{w})=\vec{x}\cdot(\vec{u}\times\vec{v})+\vec{x}\cdot(\vec{u}\times\vec{w})=\vec{x}\cdot(\vec{u}\times\vec{v}+\vec{u}\times\vec{w})$$ 
			By the cancellation property of dot product, $\vec{u}\times(\vec{v}+\vec{w})=\vec{u}\times\vec{v}+\vec{u}\times\vec{w}$. And, by similar process, we get
			$$\vec{x}\cdot((\vec{u}+\vec{v})\times\vec{w})=det(\vec{x},\vec{u}+\vec{v},\vec{w})=det(\vec{x},\vec{u},\vec{w})+det(\vec{x},\vec{v},\vec{w})=\vec{x}\cdot(\vec{u}\times\vec{w}+\vec{v}\times\vec{w})$$
			So, by the cancellation property of dot product, $(\vec{u}+\vec{v})\times\vec{w}=\vec{u}\times\vec{w}+\vec{v}\times\vec{w}$.
			\item[(iii)] $$\vec{x}\cdot((\lambda \vec{u})\times \vec{v})=det(\vec{x},\lambda\vec{u},\vec{v})=\lambda det(\vec{x},\vec{u},\vec{v})=\lambda(\vec{x}\cdot(\vec{u}\times\vec{v}))=\vec{x}\cdot(\lambda(\vec{u}\times\vec{v}))$$
			By the cancellation property of dot product, $(\lambda \vec{u})\times \vec{v}=\lambda(\vec{u}\times\vec{v})$. And,
			$$\vec{x}\cdot((\lambda \vec{u})\times \vec{v})=det(\vec{x},\lambda\vec{u},\vec{v})=\lambda det(\vec{x},\vec{u},\vec{v})=det(\vec{x},\vec{u},\lambda\vec{v})=\vec{x}\cdot(\vec{u}\times(\lambda\vec{v}))$$
			By the cancellation property of dot product, $(\lambda \vec{u})\times \vec{v}=\vec{u}\times(\lambda\vec{v})$.
		\end{itemize}
		\item[(b)] By the chain rule on the partial derivatives, we have
		$$\vec{Y}_s(s,t)=(\vec{X}\circ T)_s(s,t)=\vec{X}_u(T(s,t))u_s(s,t)+\vec{X}_v(T(s,t))v_s(s,t)$$
		And
		$$\vec{Y}_t(s,t)=(\vec{X}\circ T)_t(s,t)=\vec{X}_u(T(s,t))u_t(s,t)+\vec{X}_v(T(s,t))v_t(s,t)$$
		And we also have
		$$det(DT(s,t))=det(\bmat{u_s & u_t\\v_s & v_t})=u_sv_t-u_tv_s$$
		So,
		\begin{align*}
		N_{\vec{Y}}(s,t)&=\vec{Y}_s(s,t)\times \vec{Y}_t(s,t)\\
		&=(\vec{X}_u(T(s,t))u_s(s,t)+\vec{X}_v(T(s,t))v_s(s,t))\times (\vec{X}_u(T(s,t))u_t(s,t)+\vec{X}_v(T(s,t))v_t(s,t))\\
		&=0+\vec{X}_u(T(s,t))u_s(s,t)\times \vec{X}_v(T(s,t))v_t(s,t)+\vec{X}_v(T(s,t))v_s(s,t)\times \vec{X}_u(T(s,t))u_t(s,t)+0\\
		&=u_s(s,t)v_t(s,t)(\vec{X}_u(T(s,t))\times \vec{X}_v(T(s,t)))-u_t(s,t)v_s(s,t)(\vec{X}_u(T(s,t))\times \vec{X}_v(T(s,t)))\\
		&=(u_s(s,t)v_t(s,t)-u_t(s,t)v_s(s,t))(\vec{X}_u(T(s,t))\times \vec{X}_v(T(s,t)))\\
		&=(detDT(s,t))N_{\vec{X}}(T(s,t))
		\end{align*}
		\item[(c)] Since we know that $T$ is a $C^1$ bijective map with $T(s,t)=(u,v)$ for $(s,t)\in E$, and $detDT(s,t)\neq 0$, so $DT(s,t)$ is invertible. And we know that both $N_{\vec{Y}} (s,t)$ and $N_{\vec{X}} (u,v)$ are integrable. Then, by the change of variable theorem, we have
	\begin{align*}
		\iint_E \|N_{\vec{Y}}(s,t)\|\,dA(s,t)&=\iint_E \|(detDT(s,t))N_{\vec{X}}(T(s,t))\|\,dA(s,t)\\
		&=\iint_E \|N_{\vec{X}}(T(s,t))\||detDT(s,t)|\,dA(s,t)\\
		&=\iint_D \|N_{\vec{X}}(u,v)\|\,dA(u,v)
	\end{align*}
	\end{itemize}
\end{solution}
\newpage

\begin{problem}[Exercise 5] (Colley 3.3.20) Calculate the flow line $\vec{x}(t)$ of the vector field $\vec{F}(x,y)=-x\vec{i}+y\vec{j}$ that passes through the point $\vec{x}(0)=(2,1)$. In addition (not in the book), sketch the field and the flow line you found.
\end{problem}
\begin{solution}
\end{solution}
\newpage

\begin{problem}[Exercise 6] (Colley 3.3.25) If $\vec{x}(t)$ is a flow line of a gradient vector field $\vec{F} = \nabla f$, show that the function $G(t) = f(\vec{x}(t))$ is an increasing function of $t$.
\end{problem}
\begin{solution}
	pf: We want to determine if $G\,'(t)$ is nonnegative. We have
	$$G\,'(t)=f\,'(\vec{x}(t))\vec{x}\,'(t)=f\,'(\vec{x}(t))\nabla f(\vec{x}(t))=f\,'(\vec{x}(t))\cdot f\,'(\vec{x}(t))\geq 0$$
	So $G(t)$ is an increasing function of $t$.
\end{solution}
\newpage

\begin{problem}[Exercise 7] Suppose $\vec{F}: \mathbb{R}^n \to \mathbb{R}^n$ is a \textit{linear} vector field, meaning that there is $A\in M_{n\times n}(\mathbb{R})$ such that
	$$\vec{F}(\vec{x}) = A\vec{x} \text{ for every}\ \vec{x} \in \mathbb{R}^n.$$
	A flow line $\vec{x}(t)$ of $\vec{F}$ then satisfies
	$$\vec{x}'(t) = \vec{F}(\vec{x}(t)) = A\vec{x}(t).$$
	This problem shows that finding flow lines of linear vector fields reduces to computing eigenvalues and eigenvectors, giving an example of the key role linear algebra plays in the study of systems of differential equations, of which $\vec{x}\,'(t) = A\vec{x}(t)$ is an example.
	\begin{itemize}
		\item[(a)] Suppose $\vec{x}(t)$ is of the form $\vec{x}(t) = e^{\lambda t}\vec{v}$ where $\vec{v}$ is some nonzero constant vector in $\mathbb{R}^n$, so that $\vec{x}(t)$ looks like
		$$\vec{x}(t) = (v_1e^{\lambda t},\ldots,v_ne^{\lambda t}).$$
		Show that such an $\vec{x}(t)$ is a flow line of $\vec{F}$ if, and only if, $\vec{v}$ is an eigenvector of $A$ with eigenvalue $\lambda$.
		\item[(b)] We will take for granted the following fact from differential equations: the collection of all $\vec{x}(t)$ satisfying the equation $\vec{x}\,'(t) = A\vec{x}(t)$ forms an $n$-dimensional real vector space, and thus any $n$ linearly independent solutions will span the entire space of solutions. Using the result of (a), find a basis for the space of flow lines of the two-dimensional linear vector field given by
		$$\vec{F}(x,y) = (-3x+3y, 3x+5y).$$
		\item[(c)] Find the flow line of $\vec{F}(x,y) = (-3x+3y,3x+5y)$ passing through $(1,1)$ at time $t=0$. What happens along this flow line as $t \to \infty$?
	\end{itemize}
\end{problem}
\begin{solution}
		\begin{itemize}
		\item[(a)]($\Rightarrow$) Suppose $\vec{x}(t)$ is a flow line of $\vec{F}$. Let $t\in\mathbb{R}$. Then, $\vec{x}\,'(t)=\vec{F}(\vec{x}(t))=A\vec{x}(t)$. Since we have
			$$A\vec{x}(t)=\vec{x}\,'(t)=(\frac{\partial(v_1e^{\lambda t})}{\partial t},\ldots,\frac{\partial(v_ne^{\lambda t})}{\partial t})=(\lambda v_1e^{\lambda t},\ldots,\lambda v_ne^{\lambda t})=\lambda\vec{x}(t)$$
			So, $Ae^{\lambda t}\vec{v}=\lambda e^{\lambda t}\vec{v}$ for all $t\in\mathbb{R}$. Thus, $A\vec{v}=\lambda\vec{v}$. Since $\vec{v}\neq\vec{0}$, so $\vec{v}$ is an eigenvector for $A$ with eigenvalue $\lambda$.\\
		($\Leftarrow$) Suppose $\vec{v}$ is an eigenvector for $A$ with eigenvalue $\lambda$. Let $t\in\mathbb{R}$. Then,
		$$\vec{x}\,'(t)=(\frac{\partial(v_1e^{\lambda t})}{\partial t},\ldots,\frac{\partial(v_ne^{\lambda t})}{\partial t})=(\lambda v_1e^{\lambda t},\ldots,\lambda v_ne^{\lambda t})=\lambda e^{\lambda t}\vec{v}=Ae^{\lambda t}\vec{v}=A(\vec{x}(t))=\vec{F}(\vec{x}(t))$$
		So, $\vec{x}(t)$ is a flow line of $\vec{F}$. 
		\item[(b)] Let $A=\bmat{-3 & 3\\3 & 5}$. Then for $(x,y)\in\mathbb{R}^2,A(x,y)=(-3x+3y,3x+5y)=\vec{F}(x,y)$. Now, we will first determine the eigenvalues for $A$.
			$$0=det(\bmat{-3 & 3\\3 & 5}-\lambda I)=det(\bmat{-3-\lambda & 3\\3 & 5-\lambda})=(-3-\lambda)(5-\lambda)-9=\lambda^2-2\lambda-24=(\lambda-6)(\lambda+4)$$
		So, $\lambda=6,-4$. Next, we will determine the associated eigenvectors.
		$$E_6=null(\bmat{-9 & 3\\3 & -1})=null(\bmat{3 & -1\\0 & 0})=span(\bmat{1\\3})$$
		$$E_{-4}=null(\bmat{1 & 3\\3 & 9})=null(\bmat{1 & 3\\0 & 0})=span(\bmat{-3\\1})$$
		So, the eigenbasis is $span(\bmat{1\\3},\bmat{-3\\1})$. Thus, the basis for the spave of flow lines is $span(\bmat{1\\3}e^{6t},\bmat{-3\\1}e^{-4t})$.
		\item[(c)] Since the flow line passes through $(1,1)$ at $t=0$, so $\vec{x}(0)=(1,1)$. And since $\vec{x}(t)$ is a flow line of $\vec{F}$, so $\vec{x}(t)$ is in the space spanned by $span(\bmat{1\\3},\bmat{-3\\1})$. Let $c_1,c_2\in\mathbb{R}$. Then,
			$$(1,1)=\vec{x}(0)=(c_1\bmat{1\\3}e^{6t}+c_2\bmat{-3\\1}e^{-4t})=(c_1\bmat{1\\3}+c_2\bmat{-3\\1})=\bmat{c_1-3c_2\\3c_1+c_2}$$
		Then we get $c_1=\frac{2}{5},c_2=-\frac{1}{5}$. So, $\vec{x}(t)=\frac{2}{5}\bmat{1\\3}e^{6t}-\frac{1}{5}\bmat{-3\\1}e^{-4t}$. As $t$ goes to infinity, the first term goes to infinity while the second term goes to zero, so $\vec{x}(t)$ goes to infinity. 
	\end{itemize}
\end{solution}
\newpage

\begin{problem}[Exercise 8] (Colley 3.4.4, 3.4.11) Find the curl and divergence of each of the following vector fields:
	$$ \vec{F}(x,y,z)=z\cos(e^{y^2})\,\vec{i}+x\sqrt{z^2+1}\,\vec{j}+e^{2y}\sin(3x)\,\vec{k} \qquad \ \vec{F}(x,y,z)=y^2z\,\vec{i}+e^{xyz}\,\vec{j}+x^2y\,\vec{k}$$
\end{problem}
\begin{solution}
\end{solution}
\newpage

\begin{problem}[Exercise 9] (Colley 3.4.28bc) The Laplacian operator, denoted $\Delta$ or $\nabla^2$, is the second-order partial differential operator defined by
	$$\nabla^2f = \frac{\partial^2 f}{\partial x^2}+\frac{\partial^2 f}{\partial y^2}+\frac{\partial^2 f}{\partial z^2}.$$
	Part (a) asks you to explain why it makes sense to think of $\nabla^2$ as $\nabla \cdot \nabla$, which is just because $\nabla^2f$ is obtained by taking the divergence of the gradient of $f$, so that $\nabla^2f = \nabla\cdot(\nabla f)$. In what follows, you may use any of the results of Exercises 21 through 25 in Colley 3.4 without justification.
	\begin{itemize}
		\item[(b)] Show that if $f$ and $g$ are $C^2$ functions, then
		$$\nabla^2(fg) = f\nabla^2g+g\nabla^2f+2(\nabla f\cdot\nabla g).$$
		\item[(c)] Show that
		$$\nabla\cdot(f\nabla g-g\nabla f) = f\nabla^2g-g\nabla^2f.$$
	\end{itemize}
\end{problem}
\begin{solution}
		\begin{itemize}
		\item[(b)]
		\item[(c)]
	\end{itemize}
\end{solution}
\newpage

\begin{problem}[Exercise 10] Consider the following vector field $\vec{F}:D\to\mathbb{R}^2$ defined on the region $D=\{(x,y)\in\mathbb{R}^2\ :\ (x,y)\neq (0,0)\}$:
	$$\vec{F}(x,y) = \left(-\frac y{x^2+y^2}\ ,\ \frac x{x^2+y^2}\right).$$
	This field has curl zero at every point of $D$, and yet is not conservative on $D$. However, it is conservative over different portions of $D$. For instance, on the portion of $D$ in the first quadrant excluding the positive $y$-axis, we have
	$$\vec{F}(x,y) = \nabla\left(\arctan\Big(\frac yx\Big)\right).$$
	\begin{itemize}
		\item[(a)] Find a potential function for $\vec{F}$ on the portion of $D$ in the second quadrant excluding the negative $x$-axis which has the value $\frac\pi 2$ along the positive $y$-axis.
		\item[(b)] Find a potential function for $\vec{F}$ on the portion of $D$ in the third quadrant excluding the negative $y$-axis which has the value $\pi$ along the negative $x$-axis.
		\item[(c)] Find a potential function for $\vec{F}$ on the portion of $D$ in the fourth quadrant excluding the positive $x$-axis which has the value $\frac{3\pi}2$ along the negative $y$-axis.
		\item[(d)] Take the potential function you found in part (c), and consider its limit as $(x,y)$ approaches the $x$-axis (keeping $x$ fixed, and allowing $y\to 0-$) from the fourth quadrant. Does this limiting value agree with the value of the potential function $\displaystyle\arctan\Big(\frac yx\Big)$ for $\vec{F}$ in the portion of $D$ in the first quadrant excluding the positive $y$-axis?
	\end{itemize}
\end{problem}
\begin{solution}
	\begin{itemize}
		\item[(a)]
		\item[(b)]
		\item[(c)]
		\item[(d)]
	\end{itemize}
\end{solution}

\end{document}
