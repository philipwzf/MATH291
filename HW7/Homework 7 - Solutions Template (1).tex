\documentclass[11pt,letterpaper,cm]{nupset}
\usepackage[margin=1in]{geometry}
\usepackage{graphicx}
\usepackage{amsmath,amssymb,amsthm,amscd,graphicx,wasysym,enumerate}
\usepackage{mathrsfs}

\newtheorem{theorem}{Theorem}

\newcommand{\mb}[1]{\boldsymbol{#1}}
\newcommand{\bvec}[1]{\left[\begin{smallmatrix} #1 \end{smallmatrix}\right]}
\newcommand{\bmat}[1]{\begin{bmatrix} #1 \end{bmatrix}}

% info for header block in upper right hand corner
\name{Solutions}
\class{Math 291-3}
\assignment{Homework 7}
\duedate{May 23, 2022}

\begin{document}

\begin{problem}[Exercise 1] (Colley 6.3.1, 6.3.2) This problem has two parts.
	\begin{itemize}
		\item[(a)] Consider the line integral $\displaystyle \int_C z^2\,dx+2y\,dy+xz\,dz$.
		\begin{itemize}
			\item[(i)] Evaluate this integral, where $C$ is the line segment from $(0,0,0)$ to $(1,1,1)$.
			\item[(ii)] Evaluate this integral, where $C$ is the path from $(0,0,0)$ to $(1,1,1)$ parametrized by $\vec{x}(t) = (t, t^2, t^3),\ 0\le t\le 1$.
			\item[(iii)] Is the vector field $\vec{F} = z^2\vec{i}+2y\vec{j}+xz\vec{k}$ conservative? Why or why not?
		\end{itemize}
		\item[(b)] Let $\vec{F} = 2xy\,\vec{i}+(x^2+z^2)\,\vec{j}+2yz\vec{k}$.
		\begin{itemize}
			\item[(i)] Calculate $\displaystyle \int_C \vec{F} \cdot d\vec{s}$ where $C$ is the path parametrized by $\vec{x}(t)=(t^2,t^3,t^5), 0 \le t \le 1$.
			\item[(ii)] Calculate $\displaystyle\int_C \vec{F} \cdot d\vec{s}$ where $C$ is the straight-line path from $(0,0,0)$ to $(1,0,0)$, followed by the straight-line path from $(1,0,0)$ to $(1,1,1)$.
			\item[(iii)] Does $\vec{F}$ have path-independent line integrals? Explain your answer.
		\end{itemize}
	\end{itemize}
\end{problem}
\begin{solution}
		\begin{itemize}
		\item[(a)] 
		\begin{itemize}
			\item[(i)] 
			\item[(ii)] 
			\item[(iii)] 
		\end{itemize}
		\item[(b)] 
		\begin{itemize}
			\item[(i)] 
			\item[(ii)] 
			\item[(iii)] 
		\end{itemize}
	\end{itemize}
\end{solution}
\newpage

\begin{problem}[Exercise 2] (Colley 6.3.20, 6.3.21) Find all functions $M(x,y)$ and $N(x,y)$ so that the following vector fields are conservative on $\mathbb{R}^2$.
	\begin{itemize}
		\item[(a)] $\vec{F}=M(x,y)\vec{i}+(x\sin(y)-y\cos(x))\vec{j}$
		\item[(b)] $\vec{F}=(ye^{2x}+3x^2 e^y)\vec{i}+N(x,y)\vec{j}$
	\end{itemize}
\end{problem}
\begin{solution}
		\begin{itemize}
		\item[(a)] 
		\item[(b)] 
	\end{itemize}
\end{solution}
\newpage

\begin{problem}[Exercise 3] (Colley 6.3.26, 6.3.27, 6.3.28) Show that the following integrals are path independent (on the domain of the integrand) and evaluate them along the given oriented curve both directly and using the Fundamental Theorem of Line Integrals.
	\begin{itemize}
		\item[(a)] $\displaystyle\int_C (3x-5y)\,dx+(7y-5x)\,dy$, $C$ is the line segment from $(1,3)$ to $(5,2)$.
		\item[(b)] $\displaystyle\int_C \frac{x\,dx+y\,dy}{\sqrt{x^2+y^2}}$, $C$ is a semicircular arc of $x^2+y^2=4$ from $(2,0)$ to $(-2,0)$.
		\item[(c)] $\displaystyle\int_C (2y-3z)\,dx+(2x+z)\,dy+(y-3x)\,dz$, $C$ is the line segment from the point $(0,0,0)$ to $(0,1,1)$ followed by the line segment from the point $(0,1,1)$ to $(1,2,3)$.
	\end{itemize}
\end{problem}
\begin{solution}
	\begin{itemize}
		\item[(a)] 
		\item[(b)] 
		\item[(c)] 
	\end{itemize}
\end{solution}
\newpage

\begin{problem}[Exercise 4] Let $\vec{F}$ be the vector field
	$$(-y+ye^y)\vec{i}  + [x+xe^y(1+y)+z]\vec{j} + (y+2)\vec{k}.$$
	Compute the line integral of $\vec{F}$ over the left half of the unit circle in the $xy$-plane oriented clockwise as viewed from the positive $z$-direction. 
	
	Hint: $\vec{F}$ is not conservative, but find a way to use the Fundamental Theorem of Line Integrals anyway.
\end{problem}
\begin{solution}
\end{solution}
\newpage

\begin{problem}[Exercise 5] Suppose the continuous vector field $\vec{F} = P\,\vec{i}+Q\,\vec{j}$ on $\mathbb{R}^2$ has the property that its line integrals are path-independent, meaning that whenever $C_1$ and $C_2$ are piecewise-smooth oriented curves with the same starting point and the same ending point, we have $$\int_{C_1} P\,dx+Q\,dy = \int_{C_2} P\,dx+Q\,dy.$$
	Define the function $f: \mathbb{R}^2 \to \mathbb{R}$ by
	$$f(x,y) \stackrel{def}{=} \int_{C} P\,dx+Q\,dy,$$
	where $C$ is any oriented piecewise-smooth curve that starts at $(0,0)$ and ends at $(x,y)$. (The fact that line integrals of $\vec{F}$ are path-independent guarantees that any such path gives the same value for the integral, so that $f(x,y)$ is indeed well-defined.) Show that $\nabla f = \vec{F}$, thus showing that $\vec{F}$ is conservative. (A similar reasoning shows that any vector field field on $\vec{R}^n$ with path-independent line integrals must be conservative.)
	\medskip
	
	Hint: To determine the partial derivative $\frac{\partial f}{\partial x}$, consider the path from $(0,0)$ to $(x,y)$ consisting of the vertical line segment from $(0,0)$ to $(0,y)$ followed by the horizontal segment from $(0,y)$ to $(x,y)$. You'll need to consider a different path when computing $\frac{\partial f}{\partial y}$. The single-variable Fundamental Theorem of Calculus will be crucial.
\end{problem}
\begin{solution}
	pf: We want to show that $\nabla f = \vec{F}$, so $f_x(a,b)=P(a,b),f_y(a,b)=Q(a,b)$ for all $a,b\in\mathbb{R}^2$. Let $(x,y)\in\mathbb{R}^2$. Suppose $x\geq 0,y\geq 0$. Then we will param $C_1$ by $\vec{r}_1(t)=(0,t),0\leq t\leq y$, and we will param $C_2$ by $\vec{r}_2(t)=(t,y),0\leq t\leq x$. Then,
	\begin{align*}
		f(x,y)&=\int_{C_1\cup C_2} P\, dx+Q\, dy\\
		&=\int_{C_1}P\, dx+\int_{C_2}Q\, dy\\
		&=\int_0^y \bmat{P(0,t)\\Q(0,t)}\cdot\bmat{0\\1}\,dt+\int_0^x \bmat{P(t,y)\\Q(t,y)}\cdot\bmat{1\\0}\,dt\\
		&=\int_0^y Q(0,t)\, dt+\int_0^x P(t,y)\, dt
	\end{align*}
	Then by the Fundamental Theorem of Calculus, $f$ is differentiable with respect to $x$ at $(x,y)$ and 
	$$f_x(x,y)=(\int_0^y Q(0,t)\, dt)_x+(\int_0^x P(t,y)\, dt)_x=0+P(x,y)=P(x,y)$$
	And we also have
	$$f_y(x,y)=(\int_0^y Q(0,t)\, dt)_y+(\int_0^x P(t,y)\, dt)_y=Q(x,y)+0=Q(x,y)$$ 
	So, $\nabla f = \vec{F}$. For the case when either $x$ or $y$ or both of them is less then 0, it's exactly the same proof but changing the parametrization and the bounds for the parametrization which will also give us the same result by applying the Fundamental Theorem of Calculus. 
\end{solution}
\newpage

\begin{problem}[Exercise 6] (Colley 6.2.8) Let $\vec{F}=3xy\,\vec{i}+2x^2\,\vec{j}$ and suppose $C$ is the oriented curve consisting of the top half of the circle $(x-1)^2+y^2=1$ oriented counterclockwise, followed by the line segment from $(0,0)$ to $(0,-2)$, followed by the line segment from $(0,-2)$ to $(2,-2)$, followed by the line segment from $(2,-2)$ to $(2,0)$. Evaluate $\oint_C \vec{F} \cdot d\vec{s}$ both directly and also by means of Green's Theorem.
\end{problem}
\begin{solution}
\end{solution}
\newpage

\begin{problem}[Exercise 7] (Colley 6.2.16, 6.2.27) This problem has two parts.
	\begin{itemize}
		\item[(a)] Use Green's Theorem to find the area between the ellipse $x^2/9+y^2/4=1$ and the circle $x^2+y^2=25$.
		\item[(b)] Show that if $C$ is the boundary of any rectangular region in $\mathbb{R}^2$ (oriented counterclockwise), then
		$$\int_C (x^2y^3-3y)\,dx + x^3y^2\,dy$$
		depends only on the area of the rectangle, not on its placement in $\mathbb{R}^2$.
	\end{itemize}
\end{problem}
\begin{solution}
	\begin{itemize}
		\item[(a)] 
		\item[(b)] 
	\end{itemize}
\end{solution}
\newpage

\begin{problem}[Exercise 8] (Colley 6.2.29, 6.2.30) This problem has two parts.
	\begin{itemize}
		\item[(a)] Let $D$ be a region to which Green's theorem applies and suppose that $u(x,y)$ and $v(x,y)$ are two $C^2$ functions whose domains include $D$. Show that
		$$\iint_D \frac{\partial(u,v)}{\partial(x,y)}\,dA(x,y) = \oint_C (u\nabla v)\cdot d\vec{s}$$
		where $C = \partial D$ is oriented as in Green's Theorem.
		\item[(b)] Let $f(x,y)$ be a $C^2$ function such that
		$$\frac{\partial^2f}{\partial x^2}+\frac{\partial^2f}{\partial y^2} = 0$$
		(i.e. $f$ is \textit{harmonic}). Show that if $C$ is any piecewise-smooth simple closed curve (i.e. a curve to which Green's Theorem applies), then
		$$\oint_C \frac{\partial f}{\partial y}\,dx-\frac{\partial f}{\partial x}\,dy = 0.$$
	\end{itemize}
\end{problem}
\begin{solution}
	\begin{itemize}
		\item[(a)] By Green's Theorem, we have
		\begin{align*}
			\oint_C (u\nabla v)\cdot d\vec{s}&=\oint_{\partial D} \bmat{uv_x\\uv_y}\cdot d\vec{s}\\
			&=\iint_D curl(\bmat{uv_x\\uv_y})\, dA(x,y)\\
			&=\iint_D u_xv_y-u_yv_x\, dA(x,y)\\
			&=\iint_D det(\bmat{u_x & u_y\\v_x & v_y})\, dA(x,y)\\
			&=\iint_D \frac{\partial(u,v)}{\partial(x,y)}\, dA(x,y)
		\end{align*}
		\item[(b)] By Green's Theorem, we have
		\begin{align*}
			\oint_C \frac{\partial f}{\partial y}\,dx-\frac{\partial f}{\partial x}\,dy &=\iint_D curl(\bmat{f_y\\-f_x})\, dA(x,y)\\
			&=\iint_D -f_{xx}-f_{yy}\, dA(x,y)\\
			&=\iint_D -(f_{xx}+f_{yy})\, dA(x,y)\\
			&=\iint_D -(\frac{\partial^2f}{\partial x^2}+\frac{\partial^2f}{\partial y^2})\, dA(x,y)\\
			&=0
		\end{align*}
	\end{itemize}
\end{solution}
\newpage

\begin{problem}[Exercise 9] Consider the $1$-form
	$$\omega = \frac{(-y+x)\,dx+(x+y)\,dy}{x^2+y^2}.$$
	Determine the value of the line integral of $\omega$ over \textit{every} simple, closed piecewise-smooth curve in $\mathbb{R}^2$ which does not pass through the origin, which means that the origin does not lie on the curve itself. The value you obtain will depend on whether or not $C$ encloses the origin, and on the orientation of $C$.
\end{problem}
\begin{solution}
\end{solution}
\newpage

\begin{problem}[Exercise 10] This problem has two parts.
	\begin{itemize}
		\item[(a)] Evaluate $$\int_{S^1} (x-y^3)dx + x^3dy\ ,$$ where $S^1$ is the unit circle in $\mathbb{R}^2$ with counterclockwise orientation.
		\item[(b)] Find a function $\lambda(x,y)$ such that for every closed, piecewise-smooth oriented curve $C$ in $\mathbb{R}^2$,
		$$\int_C (x-y^3)\,dx + x^3\,dy = \int_C \lambda(x,y)\,dy.$$
	\end{itemize}
	Fun Fact: This problem appeared on one of Prof. Peterson's Ph.D. qualifying examinations.
\end{problem}
\begin{solution}
		\begin{itemize}
		\item[(a)] 
		\item[(b)] 
	\end{itemize}
\end{solution}

\end{document}
